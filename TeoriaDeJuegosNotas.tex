\documentclass{article}
\usepackage{graphicx} % Required for inserting images

\title{Parte 2: Ciencias del Comportamiento}
\author{Juani Elosegui}
\date{Octubre 2024}

\begin{document}

\maketitle

    \section*{Módulo I - Sesgos congnitivos (Parte I)}
        \textbf{Sesgo de competencia:} evitamos colaborar incluso cuando nos conviene. (Falacia de "suma cero")
        \\
        \\
        A través de experimentos y observaciones se estudia la conducta humana, y en base a eso, se puede pensar acerca de una economía del comportamiento.

        Se piensa cómo los individuos o grupos de personas \emph{toman o deberían tomar} las decisiones.
        \\
        \\
        Tres enfoques:
        \begin{itemize}
            \item Enfoque normativo: con supuestos, se asume el comportamiento de los agentes racionales: "econs".
            \item Enfoque descriptivo: cómo decidimos en la práctica.
            \item Enfoque prescriptiva: genera "recetas" acerca de cómo deberían de actuar para hacer la mejor decisión.
        \end{itemize}
        Los \textbf{agentes racionales} son los que están interesados en su beneficio propio/utilidad. Los \emph{homo economicus} conocen sus preferencias, conocen todas las alternativas y eligen siempre la mejor opción posible y que más rédito le dé.
        \\
        También, actúan de manera neutral frente al riesgo y las pérdidas, ignoran opciones irrelevantes, etc.
        \\
        \\
        En general, los humanos normales no sabemos lo que queremos hasta que lo vemos, no contemplamos todas las opciones en simultáneo, no siempre elegimos la mejor opción y tenemos un exceso de confianza en nuestros pronósticos.
        \\
        \\
        Si sos irracional no quiere decir que seas un boludo errático, si no que tu comportamiento es distinto al esperado.
        \\
        La \textbf{economía del comportamiento} incorpora nociones de psicología y neurociencias al análisis de la toma ade decisiones económicas. Esta nace cuando se demuestra la existencia de agentes irracionales.
        \\
        \\
        Para \textbf{Herbert Simon}, tenemos todos un poco de irracionalidad acotada. No tomamos decisiones óptimas pero sí tomamos decisiones que nos puedan satisfacer.
        \\
        \textbf{Daniel Kahneman} explicó la teoría prospectiva.
        \\
        \textbf{Richard Thaler} se especializó en intervenciones minimalistas que pueden pasar desapercibidas pero que cambian el comportamiento de manera sustancial.
        \\
        \\
        El efecto que generan las pérdidas suelen ser mucho mayor que el efecto que nos da el placer cuando ganamos.
        \\
        \\
        Los \textbf{sesgos cognitivos} son distorsiones sistemáticas en la racionalidad y en cómo procesamos la información las personas.
        \\
        \\
        El \textbf{sistema 1} es un tipo de pensamiento en los humanos que actúa de manera rápida, instintiva, emocional y subconsciente.
        \\
        El \textbf{sistema 2} es el otro tipo de pensamiento, que se identifica como más lento, deliberativo, lógico y consciente.

    \section*{Módulo II: Sesgos Cognitivos (Parte II)}
        El \textbf{efecto de anclaje} ocurre cuando desconocemos un número y, antes de pensar cuál es nuestro mejor estimativo, contemplamos un valor dado. El \emph{ancla} será ese valor que nos dieron, y nuestro estimativo estará sesgado por ese ancla. Si el valor del ancla es muy alto, podemos sobreestimar el valor verdadero; y si el valor del ancla es muy bajo, lo podemos subestimar.
        \\
        \\
        Si tenemos dos anclas: $a_{low}, a_{high}$ y dos medias de respuestas según cada ancla: $m_{low}, m_{high}$. Se define el \textbf{índice de anclaje} como:
        \[\frac{m_{high} - m_{low}}{a_{high} - a_{low}}\]
        \\
        \\
        Toma fuerza el efecto del anclaje en la economía porque se puede hacer lo mismo pero con los precios.
        \\
        \\
        La \textbf{coherencia arbitraria} es un fenómeno que explica que si bien los precios están influenciados comúnmente por valores arbitrarios (anclas), estas anclas son valores coherentes dentro de todo.
        \\
        \\
        El concepto de \textbf{justicia} les pone un límite a cualquier precio en las transacción. A los humanos nos preocupa que las transacciones sean justas.

    \section*{Módulo III: Manejando la Incertidumbre}
        Supongamos que tenemos un grado de información nula acerca de una decisión a tomar. Este tipo de decisiones se llama \textbf{decisiones bajo riesgo}. No sabemos en qué estado se encuentra o se encontrará el mundo, pero sabemos \emph{con qué probabilidad pasará cada cosa}.

        Una decisión bajo riesgo puede ser si una aseguradora decide asegurar a alguien, o si un inversor invierte en acciones.

    \section*{Módulo IV: Distorsionando la Incertidumbre}

    \section*{Módulo V: Rompiendo la Teoría}
        
    
\end{document}