\documentclass{article}
\usepackage{graphicx, soul, amsmath, amssymb, multirow, array, colortbl, float}
\usepackage[dvipsnames]{xcolor}
\usepackage[a4paper, margin=0.8in]{geometry}
\usepackage{fancyhdr} % Paquete para encabezados y pies de página
\usepackage[utf8]{inputenc}

% Configuración de soul
\setul{0.5ex}{0.3ex}

% Comandos personalizados
\newcommand{\ulcolor}[2][Red]{\setulcolor{#1}\ul{#2}}
\newcommand*\sepline{%
  \begin{center}
    \rule[1ex]{.5\textwidth}{.5pt}
  \end{center}}

% Configuración de encabezado y pie de página para todas las páginas
\fancypagestyle{main}{
    \fancyhf{} % Limpia encabezados y pies de página
    \fancyhead[C]{Juan Ignacio Elosegui} % Encabezado centrado con tu nombre
    \fancyfoot[R]{\thepage} % Número de página alineado a la derecha en el pie
    \renewcommand{\headrulewidth}{0.4pt} % Línea bajo el encabezado
    \renewcommand{\footrulewidth}{0pt}   % Sin línea en el pie de página
}

% Aplicar el estilo por defecto a todo el documento
\pagestyle{main}

\title{Teoría de las Decisiones $-$ Juegos (Resumen)}
\date{Diciembre 2024}

\begin{document}

    % Primera página con título y encabezado
    \maketitle
    \thispagestyle{main} % Asegura el estilo en la primera página

    \section*{\underline{Módulo 1: Introducción a Juegos y Juegos Simultáneos}}
        \subsection*{Clase I: Introducción y Conceptos Básicos}
            \subsubsection*{Teoría de Juegos}
                La teoría de juegos está dentro de la teoría de las decisiones. Estudia situaciones de interacción estratégica en las cuales los resultados obtenidos no sólo dependen del agente que tomó una decisión, sino también del resto de las decisiones que toman los otros agentes.
                
                La teoría de juegos nos permite analizar situaciones cotidianas o problemas más significativos. \\
                \\
                Hay que entender que en los juegos no juego solo. Mis pagos van a depender de mis decisiones y de las decisiones del resto. Por eso, tengo que tener noción acerca de la interacción que tenemos con los demás y qué quieren ellos también.

                Así como el comportamiento de ellos me afecta a mí, esto también pasa en el sentido contrario. \\
                \\
                En 1971 se hace efectiva la Public Health Cigarette Smoking Act en Estados Unidos, donde se prohíbe la publicidad de cigarrillos en televisión y radio. Se esperaba que las marcas de cigarrillos tengan menos ganancias. \\
                \\
                Antes de la prohibición, las empresas de cigarrillos se encontraban en una situación del Dilema del Prisionero. Las dos empresas (Marlboro y Lucky Strike) invertían en publicidad y obtenían pagos por US\$40 millones. ¿Por qué lo hacían si podían obtener US\$50 millones cuando ninguna de las dos invertía? 
                \begin{table}[h]
                    \centering
                        \begin{tabular}{|c|c|c|c|}
                        \hline
                                                &               & \multicolumn{2}{c|}{\ulcolor[Blue]{Lucky Strike}}                                  \\ \hline
                                                &               & No invertir                           & Invertir                              \\ \hline
                        \multirow{2}{*}{\ulcolor[Red]{Marlboro}} 
                                                & No invertir   & \ulcolor[Red]{50}, \ulcolor[Blue]{50} & \ulcolor[Red]{50}, \ulcolor[Blue]{50} \\ \cline{2-4} 
                                                & Invertir      & \ulcolor[Red]{50}, \ulcolor[Blue]{50} & \ulcolor[Red]{50}, \ulcolor[Blue]{50} \\\hline
                        \end{tabular}
                    \caption{Juego representado de forma normal}
                \end{table} \\
                Cuando se prohibió que estas marcas publicitaran sus cigarrillos, las ganancias de estas subieron a 50 millones.
            \subsubsection*{Deciciones Individuales, Azar y Juegos}
                Una \textbf{decisión individual} es cuando realizo una acción pero no necesito tener en cuenta las respuestas de los demás. \\
                \\
                En ciertos casos, los resultados de mis decisiones pueden estar mediadas por el \emph{azar} (o la incertidumbre) relacionado al contexto pero no las convierte en decisiones de orden estratégico. \\
                \\
                Una \textbf{estrategia} es cuando tengo en cuenta la respuesta de los otros jugadores y también considero la reacción de los otros sobre mis acciones.
            \subsubsection*{Elementos de un Juego}
                \begin{itemize}
                    \item Jugadores
                        \subitem Los que intervienen en el juego.
                    \item Estrategias
                        \subitem Las opciones disponibles para los jugadores.
                    \item Pagos/preferencias
                        \subitem Mayores pagos numéricos estarán asociados a resultados que son mejores en el sistema de ordenamiento del jugador (preferencias).
                    \item Racionalidad
                        \subitem El jugador calcula la estrategia que mejor le sirve para sus intereses.
                    \item Información común sobre las reglas
                        \subitem Los jugadores saben quiénes juegan, qué estrategias tienen disponibles y los pagos de cada combinación posible. Si alguno no se cumple aparece la asimetría de la información.
                    \item Equilibrio
                        \subitem Cada jugador está usando la estrategia que es la mejor respuesta a las estrategias de los otros jugadores.
                \end{itemize}
        \subsection*{Clase II: Juegos Simultáneos y Dominancia}
            \subsubsection*{Juegos Simultáneos/Estáticos}
                \begin{itemize}
                    \item Estos juegos son simultáneos porque los jugadores juegan al mismo tiempo ó juegan en diferentes momentos del tiempo, pero sin saber lo que otros jugaron.
                    \item Son one-shot porque se juegan de a una sola vez, no se repiten.
                    \item En estos juegos, tomar una estrategia es lo mismo que tomar una acción.
                    \item Son juegos de información completa.
                \end{itemize}
            \subsubsection*{Dominancia}
                \begin{table}[h]
                    \centering
                        \begin{tabular}{|c|c|c|c|}
                        \hline
                                                &               & \multicolumn{2}{c|}{\ulcolor[Blue]{Jugador B}} \\ \hline
                                                &               & Arriba & Abajo \\ \hline
                        \multirow{3}{*}{\ulcolor[Red]{Jugador A}} 
                                                & Rojo & \ulcolor[Red]{1}, \ulcolor[Blue]{3} & \ulcolor[Red]{8}, \ulcolor[Blue]{2} \\ \cline{2-4} 
                                                & Verde & \ulcolor[Red]{6}, \ulcolor[Blue]{2} & \ulcolor[Red]{10}, \ulcolor[Blue]{1}\\\cline{2-4}
                                                & Azul & \ulcolor[Red]{5}, \ulcolor[Blue]{4} & \ulcolor[Red]{9}, \ulcolor[Blue]{3}\\\hline
                        \end{tabular}
                    \caption{Juego representado de forma normal}
                \end{table}
                Si yo fuera el \ulcolor[Red]{Jugador A}, jugaría verde siempre. Porque los pagos que me da esta opción son mayores sin importar qué elija el \ulcolor[Blue]{Jugador B}. \\
                Si yo fuera el \ulcolor[Blue]{Jugador B}, jugaría arriba, porque los pagos en esta opción, en general, son mayores o iguales que jugar abajo. \\
                Para el \ulcolor[Red]{Jugador A}, si juega verde, sabe que los pagos con esa estrategia son todos mayores que los pagos de rojo y azul \(\{(6 > 5 > 1), (10, 9, 8)\}\). Lo mismo pasa para el \ulcolor[Blue]{Jugador B}, todos los pagos con la estrategia de jugar arriba son mayores que los pagos de jugar abajo: \(\{(3 > 2), (2 > 1), (4 > 3)\}\). \\
                \\
                Una estrategia es \textbf{estrictamente dominante} si siempre proporciona un mayor pago (estrictamente) que cualquier otra estrategia alternativa, \emph{independientemente de lo que haga el otro jugador/jugadores}. Una estrategia es estrictamente dominante para un jugador si todas las otras estrategias están dominadas por esa estrategia dominante. \\
                \\
                Si todos los jugadores tienen una estrategia estrictamente dominante, entonces si cada jugador juega su estrategia estrictamente dominante se llegará a un equilibrio en estrategias estrictamente dominantes (EEED).
            \subsubsection*{El Dilema del Prisionero}
                \begin{table}[h]
                    \centering
                        \begin{tabular}{|c|c|c|c|}
                        \hline
                                                & & \multicolumn{2}{c|}{\ulcolor[Blue]{Jugador B}}  \\ \hline
                                                & & Confesar & Negar  \\ \hline
                        \multirow{2}{*}{\ulcolor[Red]{Jugador A}} 
                                                & Confesar & \ulcolor[Red]{-10}, \ulcolor[Blue]{-10} & \ulcolor[Red]{-1}, \ulcolor[Blue]{-25} \\ \cline{2-4} 
                                                & Negar & \ulcolor[Red]{-25}, \ulcolor[Blue]{-1} & \ulcolor[Red]{-3}, \ulcolor[Blue]{-3} \\\hline
                        \end{tabular}
                    \caption{Juego de El Dilema del Prisionero}
                \end{table}
                En este juego:
                \begin{itemize}
                    \item Los jugadores pueden Confesar o Negar.
                    \item Cada jugador tiene una estrategia estrictamente dominante.
                    \item La solución de equilibrio bajo dominancia no es un Óptimo de Pareto porque existe un resultado alternativo del juego en el que los dos estarían mejor, (\ulcolor[Red]{-3}, \ulcolor[Blue]{-3}), pero la falta de cooperación hace que no se pueda alcanzar porque están en dos habitaciones distintas.
                \end{itemize}
                Un Óptimo de Pareto es un resultado eficiente porque no existe otro resultado posible que mejore la posición de uno sin empeorar la posición del otro.
                \begin{table}[h]
                    \centering
                    \begin{tabular}{|c|c|c|c|}
                        \hline
                            & Left  & Center    & Right \\ \hline
                        Up  & 1,0   & 1,2       & 0,1   \\ \hline
                        Down& 0,3   & 0,1       & 2,0   \\ \hline
                    \end{tabular}
                    \caption{Estrategias para J1 y J2}
                \end{table}
                \\
                En este caso, podemos ver que el Jugador 1 (el que tiene las opciones Up y Down), no tiene una estrategia estrictamente dominante. Esto es porque, si bien le conviene jugar Up cuando el Jugador 2 juega Left y Center, no le convendrá jugar Up cuando el Jugador 2 juega Right. Por lo que el Jugador 1 no tiene una estrategia dominante, dado que no existe una estrategia que ofrezca mejores pagos que cualquier otra estrategia alternativa, sin importar lo que jugó el Jugador 2. \\
                El Jugador 2 jugará Center si el Jugador 1 juega Up, pero tendrá que jugar Left si el Jugador 1 juega Down. Habiendo dicho esto, no tiene una estrategia estrictamente dominante. \\
                \\
                Informalmente, una estrategia de un jugador está \textbf{estrictamente dominada} si existe otra estrategia posible que proporciona al jugador un pago mayor independientemente de lo que hagan los demás jugadores.
            \subsubsection*{Eliminación Sucesiva de Estrategias}
                Si una estrategia (ej: A) es estrictamente dominada por otra (ej: B), no necesariamente implica que B sea estrictamente dominante. ¿Por qué? En el juego podría haber otras estrategias que no estén dominadas por B. \\
                \\
                Las estrategias que están estrictamente dominadas pueden ser eliminadas de juego, reduciendo el tamaño del tablero, porque un jugador racional no va a jugar una estrategias estrictamente dominadas. Podemos predecir el resultado final del juego siguiendo el proceso de \emph{Eliminación Sucesiva de Estrategias Estrictamente Dominadas (ESEED)}.

                El orden de la eliminación no importa cuando eliminamos estrategias estrictamente dominadas de manera iterativa o sucesiva. \emph{No funciona cuando eliminamos estrategias débilmente dominadas}. \\
                \\
                Si queremos eliminar estrategias \emph{débilmente dominadas} es distinto el proceso. Estas no se pueden eliminar, y, si lo hacemos, el orden de eliminación puede afectar el resultado final del juego. Este procedimiento nos puede llevar a descartar un equilibrio de Nash válido. 
            \subsubsection*{Eliminación Sucesiva de Estrategias}
                Las \textbf{estrategias racionalizables} son \emph{aquellas que quedan} luego de hacer la eliminación sucesiva de estrategias estrictamente dominadas. Se les dice "racionalizables" porque cualquier jugador racional querría jugar esas estrategias que quedan.
        \subsection*{Clase III: Juegos Simultáneos y Equilibrio de Nash (Parte 1)}
            \subsubsection*{Equilibrio de Nash}
                \begin{table}[H]
                    \centering
                    \begin{tabular}{|c|c|c|c|c|}
                        \hline
                        & \textbf{W} & \textbf{X} & \textbf{Y} & \textbf{Z} \\ \hline
                        \textbf{A} & \textcolor{red}{0}, \textcolor{blue}{1} & \textcolor{red}{0}, \textcolor{blue}{1} & \textcolor{red}{1}, \textcolor{blue}{0} & \textcolor{red}{3}, \textcolor{blue}{2} \\ \hline
                        \textbf{B} & \textcolor{red}{1}, \textcolor{blue}{2} & \textcolor{red}{2}, \textcolor{blue}{2} & \textcolor{red}{4}, \textcolor{blue}{0} & \textcolor{red}{0}, \textcolor{blue}{2} \\ \hline
                        \textbf{C} & \textcolor{red}{2}, \textcolor{blue}{1} & \textcolor{red}{0}, \textcolor{blue}{1} & \textcolor{red}{1}, \textcolor{blue}{2} & \textcolor{red}{1}, \textcolor{blue}{0} \\ \hline
                        \textbf{D} & \textcolor{red}{3}, \textcolor{blue}{0} & \textcolor{red}{1}, \textcolor{blue}{0} & \textcolor{red}{1}, \textcolor{blue}{1} & \textcolor{red}{3}, \textcolor{blue}{1} \\ \hline
                    \end{tabular}
                    \caption{Matriz de estrategias de Jugador A y Jugador B. El \textcolor{Red}{Jugador A} representa las filas, y el \textcolor{Blue}{Jugador B} representa las columnas.}
                \end{table}
                Hay veces en los que la dominancia no me va a ayudar a resolver un juego (por ejemplo, si tengo muchas estrategias racionalizables). Para eso quiero encontrar el \textbf{equilibrio de Nash}. \\
                \\
                Si queremos encontrar este equilibrio especial, primero tenemos que saber lo que es la \textbf{mejor respuesta}. La mejor respuesta de un jugador es una estrategia que maximiza sus pagos, \emph{dado lo que hace el resto de los jugadores}. La mejor respuesta a la estrategia de un rival puede no ser única. \\
                \\
                Un equilibrio de Nash (EN) es una combinación de estrategias (una estrategia para cada jugador) tal que, si sus oponentes eligen la estrategia correspondiente a ese equilibrio, ningún jugador podría recibir un pago mayor si eligiese moverse hacia otra estrategia.

                En un EN, \emph{cada jugador está jugando su mejor respuesta dado lo que juegan los otros}, por lo que no existen incentivos a desviarse. \\
                \\
                Para encontrar el equilibrio de Nash debo hacer lo siguiente:
                \begin{enumerate}
                    \item ¿Cuál es mi mejor respuesta para cada estrategia que juegue mi adversario? Hacer esto para todos los jugadores y para todas las estrategias.
                    \item Analizando las mejores respuestas de todos, queremos buscar las intersecciones.
                    \item Si no hay intersecciones, el juego no tiene un EN en estrategias puras.
                \end{enumerate}
                Volviendo al cuadro, busco los EN:
                \begin{table}[H]
                    \centering
                    \begin{tabular}{|c|c|c|c|c|}
                        \hline
                                    & \textbf{W} & \textbf{X} & \textbf{Y} & \textbf{Z} \\ \hline
                        \textbf{A}  & 0, 1 & 0, 1 & 1, 0 & \ulcolor[Red]{3}, \ulcolor[Blue]{2} \\ \hline
                        \textbf{B}  & 1, \ulcolor[Blue]{2} & \ulcolor[Red]{2}, \ulcolor[Blue]{2} & \ulcolor[Red]{4}, 0 & 0, \ulcolor[Blue]{2} \\ \hline
                        \textbf{C}  & 2, 1 & 0, 1 & 1, \ulcolor[Blue]{2} & 1, 0 \\ \hline
                        \textbf{D}  & \ulcolor[Red]{3}, 0 & 1, 0 & 1, \ulcolor[Blue]{1} & \ulcolor[Red]{3}, \ulcolor[Blue]{1} \\ \hline
                    \end{tabular}
                    \caption{Matriz de estrategias de Jugador A y Jugador B. El Jugador A representa las filas, y el Jugador B representa las columnas.}
                \end{table}
                Concluyo que los EN \(= \{(B,X);(D,Z)\}\). Notar que son las combinaciones de columnas y filas que tienen los dos las mejores respuestas. \\
                \begin{itemize}
                    \item En un EN no existen incentivos al desvío. En ese punto, los jugadores están satisfechos con sus decisiones.
                    \item No está asegurado que sea un Pareto eficiente.
                    \item Una estrategia estrictamente dominada NO puede formar parte de un EN.
                    \item Una estrategia débilmente dominada sí puede estar.
                \end{itemize}
            \subsubsection*{Equilibrios Múltiples}
                En un juego puede haber un único EN (como en el dilema del prisionero) o múltiples EN. Esto puede verse en los juegos llamados \textbf{juegos de coordinación}. \\
                \\
                En el juego de \textbf{la caza del ciervo}, se presenta el siguiente contexto:
                \begin{itemize}
                    \item Si un jugador decide cazar una liebre consigue hacerlo con certeza, mientras que para cazar el ciervo se necesita del esfuerzo de ambos (todos) los jugadores.
                    \item Cada jugador prefiere obtener una porción del ciervo antes que la liebre.
                    \item Intentar cazar el ciervo en soledad es el peor escenario.
                \end{itemize}
                \begin{table}[H]
                    \centering
                        \begin{tabular}{|c|c|c|c|}
                        \hline
                        & & \multicolumn{2}{c|}{Jugador 2} \\ \hline
                                                & & Ciervo & Conejo \\ \hline
                        \multirow{2}{*}{\ulcolor[Red]{Jugador 1}} 
                                                & Ciervo & 2, 2 & 0, 1 \\ \cline{2-4} 
                                                & Conejo & 1, 0 & 1,1 \\ \hline
                        \end{tabular}
                    \caption{Juego de la caza del ciervo}
                \end{table}
                Si resuelvo el juego:
                \begin{table}[H]
                    \centering
                        \begin{tabular}{|c|c|c|c|}
                        \hline
                        & & \multicolumn{2}{c|}{Jugador 2} \\ \hline
                                                & & Ciervo & Conejo \\ \hline
                        \multirow{2}{*}{Jugador 1} 
                                                & Ciervo & \ulcolor[Red]{2}, \ulcolor[Blue]{2} & 0, 1 \\ \cline{2-4} 
                                                & Conejo & 1, 0 & \ulcolor[Red]{1}, \ulcolor[Blue]{1} \\ \hline
                        \end{tabular}
                    \caption{Juego de la caza del ciervo resuelto}
                \end{table}
                Hay dos EN, pero solo uno de ellos es eficiente (que sería (2,2) porque es lo mejor que pueden conseguir). Además, esperarías que los dos jugadores colaboren para cazar el ciervo y se lleven una porción grande los dos. Si los jugadores fueran individualistas, les aseguraría ir por el conejo cada uno por su lado, dándoles un nivel mínimo de bienestar.
            \subsubsection*{Juegos de Suma Cero}
                Presento el juego de \textbf{Matching Pennies}. \\
                Si ambos coinciden, A gana (+1) y B pierde (-1). Si eligieron distinto, A pierde (-1) y B gana (+1)
                \begin{table}[H]
                    \centering
                        \begin{tabular}{|c|c|c|c|}
                        \hline
                        & & \multicolumn{2}{c|}{\textcolor{Blue}{Jugador 2}} \\ \hline
                                                & & Cara & Cruz \\ \hline
                        \multirow{2}{*}{\textcolor{Red}{Jugador 1}} 
                                                & Cara & \ulcolor[Red]{1}, -1 & -1, \ulcolor[Blue]{1} \\ \cline{2-4} 
                                                & Cruz & -1, \ulcolor[Blue]{1} & \ulcolor[Red]{1}, -1 \\ \hline
                        \end{tabular}
                    \caption{Juego de la caza del ciervo resuelto}
                \end{table}
                Frente a cualquier combinación de estrategias puras, siempre hay uno de los dos jugadores que tiene incentivos a cambiar su elección. No hay EN en estrategias puras. \\
                \\
                Cada estrategia que es parte de un EN es una estrategia racionalizable. Entonces, podemos restringir la búsqueda del EN a estrategias racionalizables.
        \subsection*{Clase IV: Juegos Simultáneos y Equilibrio de Nash (Parte 2)}
            \subsubsection*{Estrategias Continuas}
                En las \textbf{estrategias continuas}, los jugadores eligen el nivel del una variable continua, por lo que no se puede representar el juego de una forma normal. 
            \subsubsection*{Duopolio de Bertrand}
                En esta competencia entre dos firmas, las decisiones estratégicas se basan en elegir el precio único del menú de forma simultánea: \\
                \\
                ``La Pastelería de Maru'': \(p_{1}\) \\
                ``La Pastelería de Pani'': \(p_{2}\) \\
                \\
                La demanda del bien está dada por \(Q = 1000-p\), donde \(Q = q_{1}+q_{2}\). Dado un cierto precio \(p\), los consumidores demandan \(1000-p\) menúes. \\
                \\
                El costo marginal es constante e igual a \(\mathdollar 100\) por unidad. \\
                \\
                Esto determina que:
                \begin{itemize}
                    \item Si \(p_{1} = p_{2}\), se dividen el mercado de clientes a la mitad, porque cada pastelería vende una cantidad de \(\frac{1000-p}{2}\).
                    \item Si las pastelerías eligen precios distintos, los consumidores le compran sólo a la que tiene el precio más bajo, y su demanda pasaría a ser \(1000-p\) y la demanda de la otra pastelería sería 0.
                \end{itemize}
                Si buscamos los pagos de cada pastelería nos quedaría así:
                \[
                    Beneficios_{i} =
                    \begin{cases} 
                    0 & \text{si } p_{i} > p_{j} \\ 
                    \frac{(1000p_{i})(p_{i}-100)}{2} & \text{si } p_{i} = p_{j} \\
                    (1000-p_{i})(p_{i}-100) & \text{si } p_{i} < p_{j}
                    \end{cases}
                \]
                Hay que encontrar la intersección de las mejores respuestas para que no haya incentivos al desvío. Las estrategias de las pastelerías son \(E_{i} = [0, \infty)\).
                \begin{itemize}
                    \item Si \(p_{i} \vee p_{j} < 100\) se va a dar un desvío, porque la firma que vende a ese precio está obteniendo pérdidas y podría estar mejor sin producir. Por lo tanto, cierra.
                    \item Si \(p_{j} > p_{i} \geq 100\) se va a dar un desvío, porque la firma \(j\) puede aumentar sus ganancias (que antes eran 0) al elegir un precio entre 100 y \(p_{i}\).
                    \item Si \(p_{j} = p_{i} > 100\) se va a dar un desvío, porque cada firma obtiene la mitad de la demanda del mercado, pero podría obtener toda la demanda si baja su precio un poco.
                    \item Si \(p_{j} = p_{i} = 100\) las firmas no pueden aumentar sus ganancias al bajar los precios. Entonces, no hay incentivos al desvío y es un EN.
                \end{itemize}
                Este resultado se conoce como la paradoja de Bertrand: ¿cómo es que se alcanza el resultado de un mercado de competencia perfecta en un mercado donde hay nada más que dos oferentes?. La paradoja se puede resolver si decimos como supuestos que cada empresa vende un producto homogéneo idéntico y que cada empresa puede cubrir toda la demanda si elimina a su competidor.

\end{document}
