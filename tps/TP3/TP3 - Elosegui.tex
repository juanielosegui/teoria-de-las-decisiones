\documentclass{article}
\usepackage{graphicx, soul, amsmath, amssymb, multirow, array, colortbl, float, pgfplots}
\usepackage[dvipsnames]{xcolor}
\usepackage[a4paper, margin=0.8in]{geometry}
\usepackage{fancyhdr} % Paquete para encabezados y pies de página
\usepackage[utf8]{inputenc}
\pgfplotsset{compat=1.18}

\setul{0.5ex}{0.3ex}

\newcommand{\ulcolor}[2][Red]{\setulcolor{#1}\ul{#2}}
\newcommand*\sepline{%
  \begin{center}
    \rule[1ex]{.5\textwidth}{.5pt}
  \end{center}}

\fancypagestyle{main}{
    \fancyhf{}
    \fancyhead[C]{Juan Ignacio Elosegui}
    \fancyfoot[R]{\thepage}
    \renewcommand{\headrulewidth}{0.4pt}
    \renewcommand{\footrulewidth}{0pt}
}

\pagestyle{main}

\title{Trabajo Práctico 3 $-$ Teoría de Juegos}
\author{Juani Elosegui}
\date{Diciembre 2024}

\begin{document}
    \maketitle
    
    \section*{Problema 1}
        \subsection*{Inciso (a)}
            Esto es verdadero. El primer jugador que juega siempre es ventajoso.
        \subsection*{Inciso (b)}
            Esto es falso. No es un juego válido porque el jugador 2 sí tiene información acerca de lo que pasó en el período anterior. Lo sabrá por la cantidad de estrategias disponibles que tiene. Si el jugador 1 juega A, el jugador 2 sabe que jugó eso porque se le presentan dos opciones. Si el jugador 1 juega B, el jugador 2 sabe que jugó eso porque tiene tres opciones.
    \section*{Problema 2}
        \subsection*{(a)}   
            Lo hice en algún cuaderno.
        \subsection*{(b)}
            El jugador 2 va a aceptar una cantidad positiva. Esto es: \\
            \[m + a(m-(1-m)) \geq 0\]
            \[m + a(m-1+m) \geq 0\]
            \[m + a(2m-1) \geq 0\]
            \[m \geq -a(2m-1)\]
            \[m \geq -2ma+a\]
            \[m+2ma \geq a\]
            \[m(1+2a) \geq a\]
            \[m \geq \frac{a}{1+2a}\]
            El jugador 1 va a ofrecer \(m =\frac{a}{1+2a}\). \\
            $\therefore$ ENPS \(\{(\text{J1: }m =\frac{a}{1+2a}; \text{J2: }m \geq \frac{a}{1+2a})\}\)
        \subsection*{(c)}
            Zzz\dots

\end{document}