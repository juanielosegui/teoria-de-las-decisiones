\documentclass{article}
\usepackage{graphicx} % Required for inserting images

\title{Guía Práctica 1 - Juegos Simultáneos (Resuelto)}
\author{Juani Elosegui}
\date{Septiembre 2024}

\begin{document}
    \maketitle

    \section*{\underline{Parte 1}}
        \subsection*{Punto 1}
            \subsubsection*{a)}
                \begin{itemize}
                    \item El equilibrio por estrategias estrictamente dominantes es EEED: \{(4, 4)\}. Porque el jugador 1 va a preferir la estrategia $Y$ siempre, sin importar lo que haga el jugador 2. Para el jugador 2, siempre va a preferir la estrategia $B$. Los dos jugadores prefieren estas estrategias porque son estrictamente dominantes por sobre los demás.
                    \item Los equilibrios bajo eliminación sucesiva de estrategias estrictamente dominadas son ES-EED: \{(4, 4)\}. Es el mismo porque los jugadores no toman en consideración las estrategias que son dominadas.
                    \item El equilibrio de Nash es \{(8,8);(4,4)\}.
                \end{itemize}
            \subsubsection*{b)}
                \begin{itemize}
                    \item EEED = \{(1,1)\}
                    \item ES-EED = \{(1,1)\}
                    \item EN = \{(1,1)\}
                \end{itemize}
            \subsubsection*{c)}
                \begin{itemize}
                    \item No existe EEED.
                    \item No existe ES-EED.
                    \item EN = \{(X,A);(Y,B)\};
                \end{itemize}

        \subsection*{Punto 2}
            \subsubsection*{a)}
                EN = \{(D,L)\}. Esto es así porque se deben analizar las mejores respuestas para J1 y J2. Si se interseccionan será el equilibrio de Nash.
            \subsubsection*{b)}
                EN = \{(Z,A)\}
\end{document}
