\documentclass{article}
\usepackage{graphicx} % Required for inserting images

\title{Modelo de Primer Parcial - Juegos (Resuelto)}
\author{Juani Elosegui}
\date{Septiembre 2024}

\begin{document}
    \maketitle

    \section*{\underline{Parte A}}
        \subsection*{Punto 1}
            \subsubsection*{a)}
                Esta afirmación es \emph{falsa}. Jugar primero o jugar segundo tiene sus ventajas, pero no SIEMPRE conviene jugar segundo. Si jugás segundo, se puede aprender de los errores del otro \emph{si los hubo}. Por otro lado, si jugás primero, por ejemplo, podés incursionar primero tu reconocimiento de marca o usar los recursos de la totalidad que estén disponibles.
            \subsubsection*{b)}
                Esta afirmación es \emph{falsa}. Un contraejemplo es la forma extensiva del dilema del prisionero.
            \subsubsection*{c)}
                Esta afirmación es \emph{verdadera}, porque que una estrategia sea estrictamente dominada significa que siempre va a ofrecer mayores pagos una estrategia que no sea esa.
            \subsubsection*{d)}
                Esta afirmación es \emph{verdadera}. Que hayan tenido una larga historia de cooperación en períodos pasados puede significar que el juego en sí esté por terminar, acá los incentivos para desviarse y obtener un beneficio inmediato pueden tentar a un jugador. Por eso, es más importante que sepan los dos cooperadores que tienen varias instancias para cooperar.
        \subsection*{Punto 2}
            El conjunto de información es el estado en el que un jugador no puede distinguir en qué nodo de la forma extensiva del juego se encuentra parado. Cuando el conjunto de información es perfecto, todos los jugadores saben los movimientos de los otros en todo momento. Cuando el conjunto de información es imperfecto, este no es el caso.
    \section*{\underline{Parte B}}
        \subsection*{Punto 1}
            Le falta algo a esa frase. Yo diría "En la competencia, la ambición individual PUEDE servir al bien común."

            La ambición individual puede beneficiar a un jugador y perjudicar al resto, pero si la ambición individual del jugador casualmente lleva a pagos de un equilibrio de Nash entonces sí lleva al bien común. En un equilibrio de Nash, cada jugador está haciendo lo mejor posible dadas las acciones de los demás, lo que puede resultar en un resultado estable y mutuamente beneficioso.
\end{document}
