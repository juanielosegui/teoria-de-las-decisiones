\documentclass{article}
\usepackage{graphicx, soul, amsmath, amssymb, multirow, array, colortbl, float, pgfplots}
\usepackage[dvipsnames]{xcolor}
\usepackage[a4paper, margin=0.8in]{geometry}
\usepackage{fancyhdr} % Paquete para encabezados y pies de página
\usepackage[utf8]{inputenc}
\pgfplotsset{compat=1.18}

% Configuración de soul
\setul{0.5ex}{0.3ex}

% Comandos personalizados
\newcommand{\ulcolor}[2][Red]{\setulcolor{#1}\ul{#2}}
\newcommand*\sepline{%
  \begin{center}
    \rule[1ex]{.5\textwidth}{.5pt}
  \end{center}}

% Configuración de encabezado y pie de página para todas las páginas
\fancypagestyle{main}{
    \fancyhf{} % Limpia encabezados y pies de página
    \fancyhead[C]{Juan Ignacio Elosegui} % Encabezado centrado con tu nombre
    \fancyfoot[R]{\thepage} % Número de página alineado a la derecha en el pie
    \renewcommand{\headrulewidth}{0.4pt} % Línea bajo el encabezado
    \renewcommand{\footrulewidth}{0pt}   % Sin línea en el pie de página
}

% Aplicar el estilo por defecto a todo el documento
\pagestyle{main}

\title{Modelo de Parcial (Teoría de Juegos) $-$ Resuelto}
\author{Juani Elosegui}
\date{Diciembre 2024}

\begin{document}
    \maketitle
    
    \section*{Parte A: Teoría}
        \subsection*{Ejercicio 1}
            \begin{itemize}
                \item (a) Falso. Jugar primero tiene sus ventajas también, por lo que no es siempre conveniente.
                \item (b) Falso. El Dilema del Prisionero tiene un solo subjuego, que es el juego completo.
                \item (c) Verdadero. Ningún jugador racional jugará una estrategia estrictamente dominada.
                \item (d) Verdadero. Si bien importa que hayan colaborado en el pasado, es clave que tengan todavía períodos por jugar.
            \end{itemize}
        \subsection*{Ejercicio 2}
            Un conjunto de información muestra lo que saben los jugadores acerca de lo que se jugó en los períodos pasados y las posibles acciones a futuro.
    \section*{Parte B: Ejercicios a Desarrollar}
        \subsection*{Ejercicio 1}
            Esto no está del todo correcto, porque depende de cómo sean las interacciones entre los agentes. Si estamos en un contexto de competencia, la ambición individual sólo va a ser al bien común si nos lleva a tomar una estrategia que represente un equilibrio de Nash o un óptimo de Pareto. Además, se están asumiendo condiciones perfectas (por ejemplo, la información perfecta).
        \subsection*{Ejercicio 2}
            \subsubsection*{(a)}
                Las estrategias racionalizables son A y B para el jugador 1 y para el jugador 2.
                \begin{table}[H]
                    \begin{tabular}{|c|c|c|c|}
                        \hline
                                & A & B & C \\ \hline
                            A & 0, 0 & 3, 4 & 6, 0 \\ 
                            B & 4, 3 & 0, 0 & 0, 0 \\
                            C & 0, 6 & 0, 0 & 5, 5 \\ \hline
                    \end{tabular}
                \end{table}
            \subsubsection*{(b)}
                EN = \{(B, A); (A, B)\}.
                \begin{table}[H]
                    \begin{tabular}{|c|c|c|}
                        \hline
                            & A & B \\ \hline
                        A & 0, 0 & \ulcolor[red]{3}, \ulcolor[blue]{4} \\ 
                        B & \ulcolor[red]{4}, \ulcolor[blue]{3} & 0, 0 \\ \hline
                    \end{tabular}
                \end{table}
                \begin{table}[H]
                    \begin{tabular}{|c|c|c|}
                        \hline
                            & A \textcolor{blue}{(q)} & B \textcolor{blue}{(1-q)} \\ \hline
                        A \textcolor{red}{(p)} & 0, 0 & 3, 4 \\ 
                        B \textcolor{red}{(1-p)} & 4, 3 & 0, 0 \\ \hline
                    \end{tabular}
                \end{table}
                Planteo la situación de igualdad: \\
                \(PE_{A, J1} = PE_{B, J1}\) \\
                \(\implies 0(q) + 3(1-q) = 4(q) + 0(1-q)\) \\
                \(\implies 3(1-q) = 4(q)\) \\
                \(\implies 3-3q = 4q\) \\
                \(\implies 3 = 7q\) \\
                \(\therefore q = \frac{3}{7}\) \\
                \\
                \(PE_{A, J2} = PE_{B, J2}\) \\
                \(\implies 0(p)+3(1-p) = 4(p)+0(1-p)\) \\
                \(\implies 3(1-p) = 4p\) \\
                \(\implies 3-3p = 4p\) \\
                \(\implies 3 = 7p\) \\
                \(\implies p = \frac{3}{7}\) \\
                \\
                El equilibrio en estrategias mixtas es: \(\{(p = \frac{3}{7}, 1-p = \frac{4}{7});(q = \frac{3}{7}, 1-q = \frac{4}{7})\}\)
            \subsubsection*{(c)}
                \begin{table}[H]
                    \begin{tabular}{|c|c|c|c|}
                        \hline
                                & A & B & C \\ \hline
                            A & 0, 0 & 3, 4 & 6, 0 \\ 
                            B & 4, 3 & 0, 0 & 0, 0 \\
                            C & 0, 6 & 0, 0 & 5, 5 \\ \hline
                    \end{tabular}
                \end{table}
                El J1 coopera si: \\
                \(VP_{coop} \geq VP_{desv}\) \\
                \(\implies 5 + 3\delta \geq 6 + (4q)\delta\) \\
                \(\implies 5 + 3\delta \geq 6 + (4\frac{3}{7})\delta\) \\
                \(\implies 5 + 3\delta \geq 6 + \frac{12}{7}\delta\) \\
                \(\implies 3\delta - \frac{12}{7}\delta \geq 6 - 5\) \\
                \(\implies \frac{21-12}{7}\delta \geq 1\) \\
                \(\implies \frac{9}{7}\delta \geq 1\) \\
                \(\implies 9\delta \geq 7\) \\
                \(\implies \delta \geq \frac{7}{9}\) \\
                \(\therefore \delta \approx 0,778\) \\
                \\
                El J2 coopera si: \\
                \(VP_{coop} \geq VP_{desv}\) \\
                \(\implies 5 + 4\delta \geq 6 + (4p)\delta\) \\
                \(\implies 5 + 4\delta \geq 6 + (4\frac{3}{7})\delta\) \\
                \(\implies 4\delta - (4\frac{3}{7})\delta \geq 6 - 5\) \\
                \(\implies 4\delta - \frac{12}{7}\delta \geq 1\) \\
                \(\implies \frac{28-12}{7}\delta \geq 1\) \\
                \(\implies \frac{16}{7}\delta \geq 1\) \\
                \(\implies 16\delta \geq 7\) \\
                \(\implies \delta \geq \frac{7}{16}\) \\
                \(\therefore \delta \approx 0,438\) \\
                \\
                Para que sea sostenible el acuerdo, tomamos el valor de paciencia del más paciente, que es el J1.
        \subsection*{Ejercicio 3}
            \begin{table}[H]
                \begin{tabular}{|c|c|c|}
                    \hline
                        & C & NC \\ \hline
                    C & -5-m, -5-m & 0-m, -15 \\ 
                    NC & -15, 0-m & -1, -1 \\ \hline
                \end{tabular}
            \end{table}
            \subsubsection*{(a)}
                Para el jugador 1: \\
                \(-5-m > -15 \wedge 0-m > -1\) \\
                \(\implies -m > -10 \wedge -m > -1\) \\
                \(\therefore m < 10 \wedge m < 1\) \\
                \(\therefore m < 1\) \\
                \\
                Para el jugador 2: \\
                \(-5-m > -15 \wedge 0-m > -1\) \\
                \(\implies -m > -10 \wedge -m > -1\) \\
                \(\therefore m < 10 \wedge m < 1\) \\
                \(\therefore m < 1\) \\
        \subsection*{Ejercicio 4}
            \subsubsection*{(a)}
                Tiene tres subjuegos: el nodo padre, y los nodos que apuntan las flechas con W y Z. El resto no es considerado subjuego porque tienen un conjunto de información imperfecta.
            \subsubsection*{(b)}
                
\end{document}